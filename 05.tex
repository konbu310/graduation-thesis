\chapter{議論}
本章では、Hyper Launcherについての議論を行う。

\newpage

\section{関連研究}
Hyper Launcherを設定するにあたり、参考になった文献について述べる。

\subsubsection{macOSでディスプレイ1枚で作業する技術}
これは筆者が感銘を受けた記事\cite{onedisplay}の一つで、ホットキー型ランチャーの習得を始めるようになったきっかけでもある。記事中ではノートパソコンの小さい画面で如何に複数のアプリケーションを切り替えて操作していくかということが詳細に述べられており、そのキーの設定や思想はHyper Launcherを設計する上で大きく影響を受けた。特に、画面を10個に分割するという考えた方は、ホットキーを予め9つに制限するという方法に繋がり、とても参考になった。

\subsubsection{FALCON}
FALCON\cite{falcon}はスマートフォン向けのアプリケーションランチャーで、ユーザーが次に使用する確率の高いアプリケーションを、ユーザーのいる場所や普段の傾向といったコンテキストから予測し推薦してくれるというシステムである。予測をしてアプリケーションを並び替えて推薦するという手法は、スマートフォンのようにアプリケーションが格子状に並んでいる場合には効果的であるといえる。ホットキーと組み合わせると、一つのキーのみで複数のアプリケーションが操作できるようなインターフェースができるかもしれないと考えられる。しかし、完璧な予測モデルを作成するのは困難であり、今回は一つのキーに複数のアプリケーションを登録できるようにすることで、その機能に近づけられるよう工夫した。

\subsubsection{HotStrokes}
HotStrokes\cite{hotstrokes}はホットキーの有用性を認めつつより習得しやすく且つGUIメニューよりも効果的な手法として、トラックパッドを利用したジェスチャによるコマンド入力を提案している。確かに視覚的に記憶しやすいジェスチャをコマンドに対応させるという試みは興味深いが、トラックパッドが使える環境は限られており、ホットキーに比べ汎用性に劣ると考えられる。インターフェースの大きな変革がない限り、ホットキーの習得をサポートするような手法を取り入れるほうが効果的だと考える。

\subsubsection{IconHK}
IconHK\cite{iconhk}はツールバー上のボタンアイコンにホットキーに関する情報を埋め込むことで、ユーザーの覚えやすさを補助しようというシステムである。ホットキー型のアプリケーションランチャーの場合、画面に表示するものは基本的に何もないため、直接的に取り入れることはできなかったが、これを機にユーザーに寄り添うための機能が必要だと考え始めた。IconHKはその使いやすさとアプリケーションとしての見た目を損なわずに済むという利点を兼ね備えており、これはデスクトップ上に情報を埋め込むなどの方法で応用できるかもしれない。

\subsubsection{ExposeHK}
ExposeHK\cite{exposehk}はホットキーの使用を助けるシステムで、モディファイアキーを長押しすることで対象アプリケーションに割り当てられているホットキーをツールバー上にプレビューできるというものだ。これにより非習熟者であっても段階的にホットキーを覚えていくことが可能となる。このようなプレビュー機能については実際に使ってもらった中でも欲しいという意見があった。Electronでは長押しの実装ができず断念したが、この機能の重要性は高いと言える。プレビューの表示方法についても様々なバリエーションが考えられ、全てを羅列するだけでなく隣接するホットキーの情報を表示するなどHyper Launcherに最適な方法を模索したい。

\subsubsection{FingerArc, FingerChord}
FingerArc及びFingerChord\cite{fingerarcandfinderchord}はホットキーの習熟をサポートするシステムである。カメラを利用して指の位置やその動きを認識し、関連するホットキーを表示してくれるようになっている。これにより非習熟者でも視覚的なガイダンスを通じてマッピングを学習できるようになっている。このような習熟をサポートするシステムは数多く提案されているが、カメラやセンサを使った比較的大掛かりな手法はあまり受け入れられないのではないかと考える。まずはホットキーの便利さを実感してもらえるよう導入の敷居を下げ、そこから使いこなしていけるようにサポートするようなシステムが必要である。


\section{展望}

\subsubsection{ネイティブアプリケーション}
今回は筆者のスキルの都合上Electronを使用した開発になったが、やはりネイティブアプリケーションと比べると実装に限界を感じる部分があった。今後様々なインターフェースを試していくためには、Swift等を利用したより高度が開発が求められる。

\subsubsection{習熟のサポート}
実際に使用してもらった中で一番感じたのは、やはりソフトウェアが使用者に寄り添っていく必要があるということだ。Hyper Launcherではあまり革新的な手法を提案できなかったが、関連研究はたくさんありそれらを組み合わせて実用レベルで色々な機能を試していきたい。

\subsubsection{熟練者向けの機能}
Hyper Launcherで実装した機能のうち、その操作性を向上させるものについては一定の効果がみられた。これらは熟練者にも有用なものであり、こういった機能もより拡充させていきたい。特に慣れていけばいくほど細かい好みが表れてくるものなので、モディファイアキーの変更や複数アプリケーションの操作方法、新たなコマンドの登録等、日々活用している人たちにさらなる利便性を提供できるようにしていきたい。
\chapter{議論}
本章では、Hyper Launcherについての議論を行う。

\newpage

\section{関連研究}

\subsection{ExposeHK}
ExposeHK\cite{}はホットキーの使用を助けるシステムで、モディファイアキーを長押しすることで対象アプリケーションに割り当てられえいるホットキーをツールバー上にプレビューできるというものだ。これにより非習熟者であっても段階的にホットキーを覚えていくことが可能となる。このようなプレビュー機能については実際に使ってもらった中でも欲しいという意見があった。Electronでは長押しの実装ができなかったが、この機能の重要性は高いと言える。プレビューの表示方法についても様々なバリエーションが考えられ、全てを羅列するだけでなく隣接するホットキーの情報を表示するなどHyper Launcherに最適な方法を模索したい。

\subsection{HotStrokes}
HotStrokes\cite{}はホットキーの有用性は認めつつより習得しやすく且つGUIメニューよりも効果的な手法として、トラックパッドを利用したジェスチャによるコマンド入力を提案している。確かに視覚的に記憶しやすいジェスチャをコマンドに対応させるという試みはおもしろいが、そもそもトラックパッドの使える環境は限られており、ホットキーに比べ汎用性に劣ると考えられる。インターフェースの大きな変革がない限り、ExposeHKのようにホットキーの習得をサポートするような手法を取り入れるほうが効果的だと言える。

\section{展望}

\subsection{ネイティブアプリケーション}
今回は筆者のスキルの都合上Electronを使用した開発になったが、やはりネイティブアプリケーションと比べると実装に限界を感じる部分があった。今後様々なインターフェースを試していくためには、Swift等を利用したより高度が開発が求められる。

\subsection{習熟のサポート}
実際に使用してもらい一番感じたのは、やはりソフトウェアが使用者に寄り添っていく必要があるということだ。Hyper Launcherではあまり革新的な手法を提案できなかったが、関連研究はたくさんありそれらを組み合わせて実用レベルで色々な機能を試していきたい。

\subsection{熟練者向けの機能}
Hyper Launcherで実装した機能のうち、その操作性を向上させるものについては一定の効果がみられた。これらは熟練者にも有用なものであり、こういった機能もより拡充させていきたい。特に慣れていけばいくほど細かい好みが表れてくるものなので、モディファイアキーの変更や複数アプリケーションの操作方法、新たなコマンドの登録等、日々活用している人たちにさらなる利便性を提供できるようにしていきたい。
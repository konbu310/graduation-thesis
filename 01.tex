\chapter{序論}
\label{chap:introduction}

本章では、本研究の背景と目的、及び本論文の構成について述べる。

\newpage

\section{背景}

アプリケーションランチャーはオペレーティングシステムに標準で搭載されているだけでなく、サードパーティ製としても多くのソフトウェアが開発、公開されている。ホットキー型のランチャー以外にも異なるインターフェースを持ったランチャーが数多く存在しており、それぞれ異なった利点/欠点がある。

\subsection{アプリケーションランチャーの種類と比較}

アプリケーションランチャーには以下のような種類があり、その特徴に合わせて様々な方法でアプリケーションを起動することができる。

\subsubsection{パレット型}

画面上の一部に固定されたパレット型のエリアにアプリケーションを登録し、マウスによるクリックで起動するタイプのランチャー。macOSにおけるDockがこれにあたる。デスクトップに常駐していることから簡単にアクセスでき、誰でも使いやすいものとなっている。しかしそのエリアは限られており、多くのアプリケーションを登録しようとすると、ボタン数を増やしたりそれぞれを小さく表示したりする必要がある。基本的にその数が増えれば増えるほど操作性が低下するため、数個から数十個の頻用するアプリケーションのみを登録して使用するのが推奨される。

\subsubsection{メニュー型}

上述したDock等にメニューを設け、マウスを使用して登録したアプリケーションを起動できるようにするタイプのランチャー。ショートカットキーと組み合わせて使用されるものもある。複数のアプリケーションを一つにまとめることで場所を節約できるだけでなく、階層化によって自分の使いやすいように整理することもできる。しかし、項目や階層が増えれば目的のアプリケーションに辿り着くまでの操作ステップは増えることになる。

\subsubsection{検索型}

アプリケーションの名前を入力することで、対象のアプリケーションを起動するタイプのランチャー。macOSにおけるSpotlightがこれにあたる。検索メニューは使用するときのみ表示されるため使用していない時は場所をとらず、キーボードのみの操作で完結しているのも利点の一つである。もちろんマウスと組み合わせて使用することもできる。また大抵の場合インクリメンタルサーチが導入されているため、名前を全て覚えていなくても起動することができる。しかし汎用性が高い反面、使い時は毎回ある程度のキーボード入力が必要となってしまうというのが欠点である。

\subsubsection{ホットキー型}

単一のキーもしくは複数のキーの組み合わせを入力するだけで、登録したアプリケーションを起動できるタイプのランチャー。今回着目しているのがこのタイプである。上で挙げたものとは違い、このタイプは標準で搭載されていないことがほとんどである。したがって自分にあったサードパーティ製ソフトウェアを探す必要がある。画面上に情報を表示する必要がなく場所を取らないことに加え、一発で特定のアプリケーションを起動できるため、とても強力なランチャーである。しかし設定が面倒であったり、どのキーにどのアプリケーションを登録したのか覚えておく必要があったりと、初心者には使いにくいタイプだとされているのも事実である。これこそが標準として機能が提供されていない理由の一つだと考えられる。

\subsection{既存のホットキー型アプリケーションランチャーの問題点}

TBD

\section{本研究の目的}

既存のホットキーを利用したアプリケーションランチャーの不便を解消し、より強力で万人に使いやすいシステムを開発することが本研究の目的である。

\section{本論文の構成}

第1章では、本研究における背景と問題意識、目的について述べた。第2章では、第1章で述べた問題意識を踏まえ、新しいホットキー型アプリケーションランチャーを提案する。第3章ではシステムの実装に関して述べ、第4章では関連研究について述べる、第5章ではシステムの考察を行い、第6章では本研究を総括する.
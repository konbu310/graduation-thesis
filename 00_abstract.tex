% アブストラクト
\begin{jabstract}

ホットキー型のアプリケーションランチャーは、特定のキーの入力のみでアプリケーションを起動することができるとても強力なユーティリティソフトウェアの一種であるが、使いこなすのが難しく広く普及しているとは言い難い。
この形式のランチャーがあまり活用されていないのは、その設定の煩雑さや設定したキーの割り当てを記憶しなければならないというハードルの高さにある。これらの問題はアプリケーションの登録方法やその操作の柔軟性といったソフトウェアの工夫次第で改善できると考えられる。

「Hyper Launcher」はホットキーに対してアプリケーションを登録するという形式を採用することで、ユーザーの導入障壁を取り除き、単一のホットキーによる複数アプリケーションの操作を可能にした新しいランチャーシステムである。これにより既存の問題点が解消されるだけでなく、新しいランチャーの使い方が可能となる。
本論文ではアプリケーションランチャーの種類や特徴についてまとめた上で、「Hyper Launcher」の設計や実装、その応用例について述べ、研究の発展性について考察する。
\end{jabstract}

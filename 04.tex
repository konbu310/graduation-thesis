\chapter{実運用及び評価}
本章では、Hyper Launcherを実運用した結果および評価について述べる。

\newpage

\section{筆者の運用}
筆者は本システムを開発段階を含め半年以上利用した。日常的にホットキー型のランチャーを使用しているため覚えやすについての評価は出来ないが、その他については大きな利点を実感することができた。特に登録の作業は格段に便利になった。今まではアプリケーションを選択した後に割り当てるキーを選択する必要があったが、そのフローが一つにまとめられたことにより、余計なことを考える必要もなくなった。また一つのホットキーに対して複数のアプリケーションを登録できるため、これまで以上に柔軟な操作が可能となった。1~9に対してカテゴリーごとに分けてアプリケーションを管理し状況に応じてそれらを使い分けたり、よく使うアプリケーションの組み合わせを一つのホットキーにまとめたりと活用の幅が広がった。ただし、複数登録の操作についてはまだまだ改善の余地があると考えられる。

\section{第三者の意見}
研究室のメンバーのうちホットキー型のランチャーをあまり使ったことのない2名に実際に利用してもらった。

\subsection{アプリケーションの登録について}
アプリケーションの登録にドラッグアンドドロップを採用した点については好意的な意見が得られた。普通は面倒な作業でもHyper Launcherでは楽しく設定できるようになっていると感じられた。また事前にホットキーがしていされているということも、設定の敷居を下げている要因の一つになっていると言える。ランチャーに限らずホットキーを登録するインターフェースは面倒なものが多いので、これは大きなアドバンテージだと考えられる。

\subsection{ホットキーの覚えやすさについて}
設定自体は簡単になっていたが、覚えやすさという点ではまだまだ難しいという意見が得られた。事前にホットキーが固定されてはいるものの、そのキーとアプリケーションの対応を覚えなければならないことには変わらず、ホットキーの使用に慣れていない人にとっては依然として敷居が高いということがわかった。

\subsection{ホットキー自体の有用性について}
今回はホットキー型のランチャーをあまり使ったことのない人に試してもらうことができたが、しばらく使っていく中でその便利さを実感してもらうことができた。特にキーの入力一発で素早く起動できるという点は好評であった。しかしこれからも使っていくかとなるとまだ障壁があり、初心者でも段階的に使いやすくなっていくような仕組みが必要なのではないかと感じた。

\subsection{問題点}
その他、細かい問題点についても意見が得られた。
\begin{itemize}
  \item 複数登録している場合の操作がわかりづらい。直感的でない。
  \item 削除時の確認はいらない。不可逆の変化ではないので簡単に消せたほうが良い。
  \item ドラッグアンドドロップの移動にバグがある。
  \item コントロールキーを入力するのが難しい。macOSの場合小指で入力する必要があるため。
\end{itemize}
これ以外にも痒い所に手が届かないという指摘はいくつかあり、そういった部分については個別にカスタマイズできるような仕組みが必要であると感じた。
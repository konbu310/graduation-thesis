\chapter{実装}

本章では、第2章で述べたシステムの設計を受け、Hyper Launcherの実装について述べる。

\newpage

\section{システム構成}

Hyper Launcherはユーザーが実際に操作するためのクライアントアプリケーションと、登録したデータを保存しておくためのデータストアから構成される。構成図を図\ref{fig:system}に示す。

\begin{figure}[h]
    \begin{center}
       \fbox{\includegraphics[width=100mm]{images/system}}
    \end{center}
    \caption{システムの構成}
    \label{fig:system}
\end{figure}

\section{クライアントアプリケーション}

クライアントアプリケーションはTypeScript及びReactなどのWeb技術によって実装されており、macOSのデスクトップアプリケーションとして動作する。

\subsection{Electron}

Electron\footnote{https://electronjs.org}はHTML、JavaScript、CSSを用いてクロスプラットフォームなデスクトプアプリケーションを作成することができるソフトウェアフレームワークである。Electron自体はChromiumとNode.jsを使用しておりWeb技術のみで開発が完結するのが利点である。Hyper LauncherではElectronのglobalShortcutというAPIを利用し、Hyper Launcherが非アクティブ状態でもホットキーの入力を特定できるようにしている。アプリケーションの起動やアクティブ化なども標準の機能で実装できるため、ランチャーに適したフレームワークと言える。

\subsection{AppleScript}

Web技術のみでは実装が難しい部分についてはAppleScriptを使用した。masOSにはosascriptと呼ばれるシェルスクリプトが存在し、これを利用することでNode.jsからAppleScriptを実行することができる。AppleScriptを使用することで、起動中のアプリケーションや特定のアプリケーションがアクティブかどうかといった情報を取得することができ、これによってデスクトップアプリケーションとして十分な機能を実装することが可能となった。ただし実行する毎に子プロセスを立ち上げるため、何も考えずに使用してしまうとラグが発生してしまうこととなる。またこれはmacOSのみで使えるものであるため、Electronの利点であるクロスプラットフォーム対応はできなくなってしまった。

\section{データストア}

ユーザーが登録した情報は、Electronが定義するユーザーデータ領域(~/Library/Application Support/hyper-launcher/)にJSONファイルとして保存されている。データのやり取りが全てローカルで完結するため、オフラインでも使用できようになっている。データはホットキーの数字をキーにしアプリケーション情報の配列を値として持ったオブジェクトになっており、アプリケーション情報にはアプリケーションの名前、パス、そしてbase64エンコードされたアイコンの3つが含まれている。これにより単一のキーに対して複数のアプリケーションを登録することが可能となっている。以下にその例を示す。ただしiconのbase64文字列と中間部分は長くなるため省略した。

\begin{lstlisting}[caption=config.json]
{
  "shortcut": {
    "1": [
      {
        "name": "ForkLift",
        "path": "/Applications/ForkLift.app",
        "icon": "***************************"
      }
    ],
    "2": [
      {
        "name": "Hyper",
        "path": "/Applications/Hyper.app",
        "icon": "***************************"
      }
    ],
    
    ・・・
    
    "9": [
      {
        "name": "Hyper Launcher",
        "path": "/Applications/Hyper Launcher.app",
        "icon": "***************************"
      }
    ],
  }
}
\end{lstlisting}
\chapter{実装}

本章では、第2章で述べたシステムの設計を受け、Hyper Launcherの実装について述べる。

\section{システムの構成}

Hyper Launcherはユーザーが実際に操作するためのクライアントアプリケーションと、登録したデータを保存しておくためのデータストアから構成される。構成図を図に示す。

\section{クライアントアプリケーション}

クライアントアプリケーションはTypeScript及びReactなどのWeb技術によって実装されており、macOSのデスクトップアプリケーションとして動作する。

\subsection{AppleScript}

各アプリケーションの起動状態を確認するなど、Web技術のみでは難しい部分についてはAppleScriptによって実装した。masOSにはosascriptと呼ばれるシェルスクリプトが存在し、これを利用することでNode.jsからAppleScriptを呼び出すことができる。これによってデスクトップアプリケーションとして十分な機能を実装することができた。

\subsection{Electron}

ElectronはChromiumとNode.jsを使用してWeb技術でWindows、Linux、macOSに対応したデスクトップアプリケーションを作成することができるソフトウェアフレームワークである。グローバルなキーイベントのハンドリングを行えることに加え、アプリケーションの起動やアクティブ化まで実装できるため、ランチャーを作るにあたってとても有用なフレームワークである。

\section{データストア}

ユーザーが登録した情報を保存するためのデータストアは、Hyper Launcherのユーザーデータ領域(~/Library/Application Support/hyper-launcher/)にJSONファイルとして永続化されている。こうすることでデータのやり取りが全てローカルで完結するため、オフラインでも使用できようになっている。なお、データの構造は以下の通りである。

\begin{lstlisting}[caption=config.json]
{
  shortcut: {
    "1": [
      {
        name: "アプリケーションの名前",
        path: "アプリケーションのパス",
        icon: "base64文字列にエンコードされたアプリケーションアイコン"
      }
    ],
    2: [],
    3: [],
    4: [],
    5: [],
    6: [],
    7: [],
    8: [],
    9: []
  }
}
\end{lstlisting}